\documentclass[a4paper,10pt]{letter}
\usepackage[utf8]{inputenc}
\usepackage{amsmath}
\usepackage{hyperref}
\usepackage[a4paper, margin=0.9in]{geometry} % Adjust margins to allow more text

\begin{document}

\begin{letter}{Admissions Committee\\
PhD Fellowship Program in Quantum Algorithms and Quantum Error Correction\\
University of Copenhagen\\
Copenhagen, Denmark}

\address{
Erick Jesús Ríos González\\
Toltecas 480, Ajusco, Coyoacán\\
Mexico City, 04300\\
\href{mailto:erickjesusriosgonzalez@gmail.com}{erickjesusriosgonzalez@gmail.com}\\
(+52) 951 587 1972\\
}

\opening{Dear Members of the Admissions Committee,}

I am writing to express my enthusiasm for the PhD fellowship in Quantum Algorithms and Quantum Error Correction at the University of Copenhagen. As I near the completion of my Bachelor’s degree in Mathematics at the Universidad Nacional Autónoma de México, complemented by a Bachelor’s degree in Physics from the Universidad Autónoma Metropolitana, I firmly believe that I am a compelling candidate for your program, given my academic rigor, unwavering dedication, and profound scientific curiosity.

Throughout my academic journey, I have cultivated a robust understanding of essential mathematical and physical concepts, including linear algebra, probability theory, statistics, differential geometry, network theory, and graph theory, alongside a solid foundation in quantum mechanics. This comprehensive knowledge has empowered me to grasp the theoretical underpinnings necessary for exploring quantum algorithms. Furthermore, my elective coursework in computer science—covering computational complexity, algorithm analysis, parallel programming, and neural networks—has equipped me with the computational proficiency required to address intricate challenges in quantum information science.

My fervor for research has driven me to engage in various projects, two of which are currently in preparation for publication. These endeavors focus on simulating and analyzing solid-state systems and quantum algorithms employing the Quantum Fourier Transform through Pennylane, which has significantly sharpened my problem-solving capabilities. Through these experiences, I have honed my programming skills in languages such as Python and C++, which I utilized to implement algorithms for diverse simulations. This background positions me to contribute effectively to your research, particularly in algorithm development.

I am particularly drawn to the University of Copenhagen due to its stellar reputation in quantum information science, especially in the domains of error correction and algorithms. The innovative research conducted by your faculty in fault-tolerant quantum computing aligns seamlessly with my long-term aspirations to contribute to the advancement of scalable and efficient quantum algorithms that mitigate the impacts of noise and decoherence. The prospect of collaborating with leading experts in the field is profoundly inspiring, and I am confident that the interdisciplinary and collaborative ethos at your institution will enable me to realize my full potential as a researcher.

Despite not yet holding a Master's degree, I have consistently demonstrated my capacity to excel in demanding academic environments and to contribute meaningfully to research initiatives. I am eager to bring my skills, determination, and passion for quantum computing to your program, where I can further cultivate my research capabilities and make significant contributions to the future of quantum technologies.

Thank you for considering my application. I look forward to the opportunity to discuss how my background and research interests align with the objectives of the PhD fellowship at the University of Copenhagen.

Sincerely,

\closing{Erick Jesús Ríos González}
\end{letter}
\end{document}