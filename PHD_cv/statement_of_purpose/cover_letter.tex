\documentclass[a4paper,10pt]{letter}
\usepackage[utf8]{inputenc}
\usepackage{amsmath}
\usepackage{hyperref}
\usepackage[a4paper, margin=0.9in]{geometry} % Adjust margins to allow more text

\begin{document}

\begin{letter}{Admissions Committee\\
PhD Fellowship Program in Quantum Algorithms and Quantum Error Correction\\
University of Copenhagen\\
Copenhagen, Denmark}

\address{
Erick Jesús Ríos González\\
Toltecas 480, Ajusco, Coyoacán\\
Mexico City, 04300\\
\href{mailto:erickjesusriosgonzalez@gmail.com}{erickjesusriosgonzalez@gmail.com}\\
(+52) 951 587 1972\\
}

\opening{Dear Members of the Admissions Committee,}

I am writing to apply for the PhD fellowship in Quantum Algorithms and Quantum Error Correction at the University of Copenhagen. Although I am completing my Bachelor’s degree in Mathematics 
at the Universidad Nacional Autónoma de México and hold a Bachelor’s degree in Physics from the Universidad Autónoma Metropolitana, I believe that my extensive research experience, academic 
preparation, and dedication to quantum computing make me a strong candidate for your program.

During my studies, I have developed a deep understanding of key mathematical and physical concepts such as linear algebra, probability theory, statistics, differential geometry, network theory, 
and graph theory and a decent background in quantum mechanics. This have allowed me to grasp the theoretical foundation required for understanding quantum algorithms. Furthermore, elective 
courses in computer science, including computational complexity, algorithm analysis, parallel programming, and neural networks, have given me the computational tools 
to tackle complex problems in quantum information science.

My passion for research has led me to actively participate in various projects, two of which are in preparation for publication. These projects involve the simulation and analysis of solid-state systems and quantum algorithms
using the Quantum Fourier Transform implemented in Pennylane, enhancing my problem-solving skills. Through these experiences, I have developed strong programming skills in languages such as Python, C++, 
which I used to implement algorithms for different simulations. This background equips me to contribute effectively to your research, particularly in the fields algorithm development.

What draws me to the University of Copenhagen is its outstanding reputation in quantum information science, particularly in the areas of error correction and algorithms. Your faculty’s innovative 
research in fault-tolerant quantum computing aligns perfectly with my long-term goals of contributing to the development of scalable, efficient quantum algorithms that mitigate the effects of noise 
and decoherence. The opportunity to work alongside leading experts in the field is incredibly motivating, and I am confident that the interdisciplinary and collaborative environment at your institution 
will help me achieve my full potential as a researcher.

Despite not yet holding a Master's degree, I have consistently demonstrated my ability to excel in demanding academic environments and contribute meaningfully to research efforts. I am eager to bring my 
skills, determination, and passion for quantum computing to your program, where I can further develop as a researcher and make significant contributions to the future of quantum technologies.

Thank you for considering my application. I look forward to the opportunity to discuss how my background and research interests align with the goals of the PhD fellowship at the University of Copenhagen.

\closing{Sincerely,\\[2em] % Space for signature}
Erick Jesús Ríos González}
\end{letter}
\end{document}