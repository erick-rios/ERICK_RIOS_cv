\documentclass[10pt]{article}
\usepackage{graphicx} % Required for inserting images
\setlength{\parindent}{0pt}
\usepackage{hyperref}
\usepackage{enumitem}
\usepackage[utf8]{inputenc} 
\usepackage[T1]{fontenc}
\usepackage[brazil]{babel}
\usepackage{lipsum}
\usepackage[left=1.06cm,top=0.98cm,right=1.06cm,bottom=0.49cm]{geometry}
\usepackage{xcolor}
\begin{document}
\begin{center}
    \textbf{\textcolor{blue}{ERICK JESUS RIOS GONZALEZ}}\\ 
    \hrulefill
\end{center}

\begin{center}
    Coyoacán \textbullet \ CDMX, 04300 \textbullet \ erickrg2700@ciencias.unam.mx \textbullet \ (+52) 951 587 1972
\end{center}

\vspace{0.5pt}
\begin{center}
    \textbf{\textcolor{blue}{Summary}}
\end{center}
\vspace{12pt}
\textbf{Erick Jesús Ríos González} is a physicist (B.Sc. in Physics, Universidad Autónoma Metropolitana) and soon–to–be graduate in Mathematics (B.Sc., Universidad Nacional Autónoma de México). With a strong background in physics, parallel computing, and mathematical modeling. He has developed expertise in Python and CUDA programming, particularly in the implementation of algorithms for high–performance computing, including parallel kernel design and optimization. His work spans both theoretical and computational aspects of quantum mechanics, quantum information theory, and numerical methods, including quantum Monte Carlo simulations techniques.

Erick has contributed to research in the areas of quantum computation and high–performance computing, and he has been involved in teaching undergraduates in CUDA–based parallel programming. He has presented at national and international conferences. He is preparing several manuscripts for peer–reviewed journals. Erick is dedicated to advancing the field of quantum computing and is eager to contribute to cutting–edge research at the University of Copenhagen.
\begin{center}
    \textbf{\textcolor{blue}{Education}}
\end{center}
\textbf{Universidad Autónoma Metropolitana} \hfill CDMX, Mexico

B.Sc. Physics

GPA 3.72/4.00 \hfill Oct 2023

Final Project: ``Study of the Phase Transition in a 2D Ising Model Using Topological Variables''

Advisor(s): Dr.\ Angel\ Alejandro\ García\ Chung, Dr.\ Marco\ Antonio\ Maceda\ Santamaría.

\vspace{12pt}

\textbf{Universidad Nacional Autónoma de México} \hfill CDMX, Mexico

B.Sc. Mathematics  \hfill Dec 2024

GPA 3.80/4.00

Thesis: ``Analysis of Correlation between Physical, Topological and Geometric Variables Present  in the Phase Transition of a 2D Ising Model with Zero External Magnetic Field''

Advisor(s): Dr.\ Angel\ Alejandro\ García\ Chung

\vspace{12pt}
\begin{center}
    \textbf{\textcolor{blue}{Employment}}
\end{center}
\vspace{12pt}
\textbf{Faculty of Sciences, UNAM} \hfill CDMX, Mexico

\textbf{Teaching Assistant} \hfill Aug 2024 – Present
\begin{itemize}[noitemsep, topsep=0pt, partopsep=0pt, parsep=0pt]
    \item Created engaging educational materials and interactive lesson plans to foster student learning and enhance comprehension.
    \item Guided students in mastering High Performance Computing techniques, empowering them to develop robust algorithms and innovative solutions.
    \item Designed and optimized CUDA scripts for various optimization problems, significantly improving algorithm assessment efficiency and enhancing students' practical coding skills through collaborative projects.
\end{itemize}

\vspace{12pt}


\textbf{Didi Chuxing Technology Co.} \hfill CDMX, Mexico

\textbf{Data Science Intern} \hfill Jun 2023 – Oct 2023
\begin{itemize}[noitemsep, topsep=0pt, partopsep=0pt, parsep=0pt]
    \item Developed advanced Python algorithms to develop robust data models for Dark Kitchen DiDi in LATAM; facilitated strategic decisions that boosted operational efficiency by 30\% and reduced data processing time by 50\%
    \item Built \texttt{Power BI} dashboards, focusing on data quality to highlight key performance metrics and actionable insights.
    \item Enhanced data accuracy by 25\% through rigorous data cleaning and validation processes.
    \item Collaborated with cross–functional teams to integrate data sources and optimize reporting workflows.
\end{itemize}

\vspace{12pt}
\begin{center}
    \textbf{\textcolor{blue}{Research Experience}}
\end{center}

\vspace{12pt}



\textbf{Universidad Autónoma Metropolitana} \hfill CDMX, Mexico

\textbf{Assistant Researcher} \hfill May 2021 – Oct 2023
\begin{itemize}[noitemsep, topsep=0pt, partopsep=0pt, parsep=0pt]
    \item Designed of object–oriented code to create a database for the 2D Ising Model.
    \item Implemented parallel computing techniques, achieving a 33\% increase in code efficiency.
    \item Leveraged high–performance computing patterns using Max Planck Institute resources, enhancing computational efficiency by 40\% and accelerating mathematical simulations by 25\%, leading to groundbreaking research advancements in theoretical mathematics.
    \item Executed complex data analysis leveraging PySpark and Matplotlib, uncovering insights that improved data processing efficiency by 30\%.
    \item Created Convolutional Neural Network (CNN) for predictions, achieving an accuracy of 93\%.
\end{itemize}

\vspace{12pt}

\textbf{Instituto Nacional de Óptica y Electrónica} \hfill Puebla, Mexico

\textbf{Assistant Researcher} \hfill Jun 2022 – Jul 2022
\begin{itemize}[noitemsep, topsep=0pt, partopsep=0pt, parsep=0pt]
    \item Developed parallel computing scripts using \textit{Python} to optimize simulations.
    \item Visualized data from simulations of HII region expansion, providing insights into astrophysical processes.
    \item Presented research results and data analysis to an audience of researchers and experts.
\end{itemize}

\vspace{12pt}

\vspace{12pt}
\begin{center}
    \textbf{\textcolor{blue}{International Schools}}
\end{center}
\vspace{12pt}

\textbf{International Centre for Theoretical Physics } \hfill São Paulo, Brazil 

\textbf{(South America Institute for Fundamental Research)}

\textbf{Assistant} \hfill Oct 2024
\begin{itemize}[noitemsep, topsep=0pt, partopsep=0pt, parsep=0pt]
    \item Quantum algorithms (Grover, Shor, VQE, QAOA), quantum simulations using Qiskit, quantum information processing, quantum noise models, quantum machine learning applications.
    \item Development of a quantum algorithm simulator using \textit{Qiskit}, implementation of matrix decomposition in a quantum simulator, analysis of quantum optimization algorithms in noisy environments.
    \item Advanced proficiency with the \textit{Qiskit} framework, implementation of quantum gates, optimization of quantum algorithms, problem–solving in quantum computing applied to optimization and machine learning.
\end{itemize}

\vspace{12pt}

\vspace{12pt}

\textbf{Abdus Salam International Centre for Theoretical Physics} \hfill Trieste, Italy

\textbf{Assistant} \hfill May 2024
\begin{itemize}[noitemsep, topsep=0pt, partopsep=0pt, parsep=0pt]
    \item Implemented advanced topics in Mathematics for machine learning, improving training efficiency on complex data by 40\%.
    \item Analyzed large amounts of data using Topological Data Analysis techniques.
    \item Engineered advanced high–dimensional statistical Python code to assess machine learning algorithms, reducing computational time by 40\% and increasing algorithm evaluation accuracy by 25\% in a team of 4 data scientists.
\end{itemize}

\vspace{12pt}

\vspace{12pt}
\begin{center}
    \textbf{\textcolor{blue}{Congress Participations}}
\end{center}
\vspace{12pt}

\textbf{–} XIII Mexican School on Gravitation and Mathematical Physics \hfill Nov – 2021

\textit{Participated in workshops and lectures on advanced topics in gravitation and mathematical physics.}

\textbf{–} LXV Congreso Nacional de Física \hfill Oct – 2022

\textit{Presented a poster on the study of the phase transition in a 2D Ising Model using topological variables. Solid–state physics, network theory and topology.}

\textbf{–} Artificial Intelligence Macrotraining Workshop \hfill Nov – 2022

\textit{Attended intensivetraining sessions on artifical intelligence and large-scale machine learning techniques.}

\textbf{–} Second School of Quantum Computing, Institute of Nuclear Sciences.\ \hfill Jul – 2023/ Aug – 2023

\textit{Completed a series of lectures and hans-on sessions on quanutm algorithms and programming using Qiskit at Institute of Nucelar Sciences of National Autonomous Univesitiy of Mexico.}

\textbf{–} Polariton School Quantum Fluids of Light Summer School \hfill

\textit{Participated in lectures on polaritons and quantum fluids of light, including experimental and theoretical approaches.}

\vspace{12pt}

\begin{center}
    \textbf{\textcolor{blue}{Projects}}
\end{center}
\vspace{12pt}
\textbf{ISINGenerator} \hfill \href{https://github.com/erick-rios/ISINGenerator}{Github}

Open–source library facilitating advanced analysis of energy, magnetization, and topological domains in a 2D Ising Model; enhanced simulation accuracy by 50\% and received 2+ GitHub stars within 3 months

\vspace{12pt}

\vspace{12pt}
\begin{center}
    \textbf{\textcolor{blue}{Skills}}
\end{center}
\vspace{12pt}
\textbf{Technical:} C/C++, Java, Python, FORTRAN90, SQL, Scala, MATLAB, R, SAS, Power BI

\textbf{Language:} Spanish(Native), English(C1), Japanese(A5)

\textbf{Laboratory:} High Performance Computing (HPC) tools, Convolutional Neural Networks (CNN), Parallel computing techniques, Artificial Intelligence development.

\textbf{Softskills:} Teamwork, Problem–solving, Communication, Leadership, Time Management.

\vspace{12pt}
\begin{center}
    \textbf{\textcolor{blue}{Teaching}}
\end{center}
\vspace{12pt}

\textbf{Seminary on Computer Science A} \hfill CDMX, Mexico

Teaching Assistant

Universidad Nacional Autónoma de México 

\textit{Introduction to Parallel Computing with MPI, OpenMP and CUDA} \hfill Aug 2024 – Dec 2024

\vspace{12pt}
\begin{center}
    \textbf{\textcolor{blue}{Articles (In Preparation)}}
\end{center}
\vspace{12pt}

\textbf{[1]} Angel García Chung, Erick Ríos González, \textit{“How to Add Two Integers in a Quantum Computing? Applications of Quantum Fouirer Transform”}. \\
\textbf{[2]} Erick Ríos González, Marisol Bermúdez Montaña, Angel García Chung et.\ al. ``\textit{Analysis of the Correlation of Physical, Topological, and Geometric Variables Present in the Phase Transition of a 2D Ising Model with Zero External Magnetic Field}
''. \\


\end{document}
